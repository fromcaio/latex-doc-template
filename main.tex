%% ===================================================================================
%% Undergraduate Thesis Template using abntex2
%% Version: 1.0
%% Based on the original abntex2 model by the abnTeX2 group and derivatives.
%% This template is designed to be a clean, modern, and easy-to-use starting
%% point for undergraduate theses, following ABNT standards.
%% ===================================================================================

% Use the new 'freelance-doc' class
\documentclass[
	% -- Memoir Class Options --
	12pt,% Font size
	openright,% Chapters start on odd-numbered pages
	oneside,% For single-sided printing
	a4paper,% Paper size
	% -- Babel Package Options --
	english,% Additional language for hyphenation
	brazil% Main document language
]{freelance-doc}

%% ===================================================================================
%% PACKAGE IMPORTS
%% ===================================================================================

% -----------------------------------------------------------------------------------
% -- Fundamental Packages
% -----------------------------------------------------------------------------------
\usepackage[T1]{fontenc}      % Use 8-bit T1 fonts
\usepackage[utf8]{inputenc}   % UTF-8 input encoding for accents and special characters
\usepackage{lmodern}          % Use the Latin Modern font family for better PDF rendering
\usepackage{indentfirst}      % Indent the first paragraph of each section
\usepackage{cmap}             % Make the PDF searchable and copyable

% -----------------------------------------------------------------------------------
% -- Graphics and Colors
% -----------------------------------------------------------------------------------
\usepackage{graphicx}         % For including images
% \usepackage{xcolor}           % Already being added
\usepackage{subcaption}       % For creating subfigures within a figure environment

% -----------------------------------------------------------------------------------
% -- Tables and Listings
% -----------------------------------------------------------------------------------
\usepackage{booktabs}         % For professional-quality tables
\usepackage{listings}         % For basic code listings
\usepackage{minted}           % For advanced, syntax-highlighted code listings (requires Python's Pygments)
% \usepackage[table]{xcolor}    % To add color to table rows/cells
\usepackage{multirow}         % For 
% -----------------------------------------------------------------------------------
% -- Mathematics
% -----------------------------------------------------------------------------------
\usepackage{amsmath, amsfonts, amssymb} % For advanced math typesetting

% -----------------------------------------------------------------------------------
% -- Code Formating
% --------------------------------------------------------------------------------------
\usepackage{inconsolata} % or try \usepackage{inconsolata} or \usepackage{FiraMono}
\lstset{
    numbers=left,
    numberstyle=\tiny,
    numbersep=8pt,                  % Space between line numbers and code
    frame=single,                   % Box around code
    framesep=6pt,                   % Space between frame and code
    xleftmargin=2em,                % Indent code block to fit line numbers
    xrightmargin=1em,
    breaklines=true,
    postbreak=\mbox{$\hookrightarrow$\space},
    basicstyle=\linespread{1.05}\ttfamily\footnotesize, % Use better mono font and size
    keywordstyle=,
    commentstyle=,
    stringstyle=,
    showstringspaces=false,
    tabsize=2,
    captionpos=b,
    rulecolor=,
}
% -----------------------------------------------------------------------------------
% -- Citation and Bibliography
% -----------------------------------------------------------------------------------
\usepackage[num]{abntex2cite}  % ABNT compliant citation style (numeric)
% For author-date style, use: \usepackage[alf]{abntex2cite}

%% ===================================================================================
%% CUSTOM COMMANDS AND CONFIGURATIONS
%% ===================================================================================

%% ===================================================================================
%% DOCUMENT INFORMATION (Updated for Freelance Work)
%% ===================================================================================

% [TODO: Fill in your project information here]
\titulo{Project Title Here}
\autor{Your Full Name (Developer)}
\cliente{Client Name or Company}
\local{City, State}
\data{\the\year}

% --- Images for the Cover Page ---
% [TODO: Make sure these image files exist in the 'imgs' folder]
\toplogo{imgs/cover-top.png}
\bottomlogo{imgs/cover-bottom.png}
\closingimage{imgs/closing-image.png} % <-- ADD THIS LINE
% -----------------------------------------------------------------------------------
% -- PDF Hyperlinks and Metadata Configuration
% -----------------------------------------------------------------------------------
\definecolor{hyperlinkblue}{RGB}{41, 5, 195} % Custom blue for links

\makeatletter
\hypersetup{
    pdftitle={\@title},
    pdfauthor={\@author},
    pdfsubject={\imprimirpreambulo},
    pdfcreator={LaTeX with abnTeX2},
    pdfkeywords={keyword1}{keyword2}{keyword3},
    colorlinks=true,
    linkcolor=hyperlinkblue,
    citecolor=hyperlinkblue,
    filecolor=magenta,
    urlcolor=hyperlinkblue,
    bookmarksdepth=4,
    bookmarkstype=toc
}
\makeatother

% -----------------------------------------------------------------------------------
% -- Spacing and Layout
% -----------------------------------------------------------------------------------
\setlength{\parindent}{1.5cm} % Paragraph indentation
\setlength{\parskip}{0.2cm}   % Vertical space between paragraphs

% -----------------------------------------------------------------------------------
% -- Indexing
% -----------------------------------------------------------------------------------
\makeindex % Generate the index file

%% ===================================================================================
%% DOCUMENT START
%% ===================================================================================
\begin{document}

% Use Brazilian Portuguese hyphenation patterns and spacing rules
\frenchspacing

% ----------------------------------------------------------
% -- PRE-TEXTUAL ELEMENTS
% ----------------------------------------------------------
\pretextual

% --- Cover and Title Page ---
% \imprimircapa
\imprimircapalogo
\imprimirfolhaderosto

% --- Other pre-textual elements (Approval page, Dedication, etc.) are in a separate file ---
% [TODO: Edit the 'pretextual_elements.tex' file with your information]
%% ===================================================================================
%% Pre-Textual Elements (pretextual_elements.tex)
%% Version: 2.0 (Freelance)
%%
%% This file contains the Executive Summary for the project deliverable.
%% ===================================================================================


% -----------------------------------------------------------------------------------
% -- EXECUTIVE SUMMARY (Sumário Executivo)
% -----------------------------------------------------------------------------------
% This section provides a high-level overview of the project document.
% -----------------------------------------------------------------------------------
\begin{resumo}[Resumo]

  % [TODO: Write the executive summary of your project here.]
  This summary should briefly describe the project's objectives, the scope of the work delivered, the key outcomes, and any major conclusions or recommendations. It is intended for stakeholders who need a quick understanding of the document's contents without reading it in its entirety. It should be written in a single paragraph.

  \vspace{\onelineskip} % Adds a blank line before the keywords
  \noindent % Prevents indentation for the keywords line
  \textbf{Palavras-chave}: tecnologia-chave; entregável-principal; nome-do-projeto.
\end{resumo}


% -----------------------------------------------------------------------------------
% -- The following academic sections have been removed for the freelance template:
% -- Folha de Aprovação (Approval Sheet)
% -- Dedicatória (Dedication)
% -- Agradecimentos (Acknowledgements)
% -- Epígrafe (Epigraph)
% -- Abstract (Second language abstract)
% -----------------------------------------------------------------------------------

% --- List of Figures ---
\pdfbookmark[0]{\listfigurename}{lof}
\listoffigures*
\cleardoublepage

% --- List of Tables (Optional) ---
% [TODO: Uncomment the following lines if your thesis contains tables]
\pdfbookmark[0]{\listtablename}{lot}
\listoftables*
\cleardoublepage

% --- List of Acronyms and Abbreviations ---
% [TODO: Add your acronyms to the 'acronyms.tex' file and uncomment below]
%% ===================================================================================
%% Acronyms and Abbreviations (acronyms.tex)
%%
%% This file contains the list of acronyms used in the thesis.
%%
%% To use this file:
%% 1. Add your acronyms below following the example format.
%% 2. In main.tex, uncomment the line that reads: %% ===================================================================================
%% Acronyms and Abbreviations (acronyms.tex)
%%
%% This file contains the list of acronyms used in the thesis.
%%
%% To use this file:
%% 1. Add your acronyms below following the example format.
%% 2. In main.tex, uncomment the line that reads: %% ===================================================================================
%% Acronyms and Abbreviations (acronyms.tex)
%%
%% This file contains the list of acronyms used in the thesis.
%%
%% To use this file:
%% 1. Add your acronyms below following the example format.
%% 2. In main.tex, uncomment the line that reads: \include{acronyms}
%% ===================================================================================

% The 'siglas' environment creates the list of acronyms.
\begin{siglas}
	% [TODO: Add your project's acronyms and abbreviations here.]
	% The format is: \item[ACRONYM] Full Name / Description
	
	\item[ABNT] Associação Brasileira de Normas Técnicas
	\item[API] Application Programming Interface
	\item[CPU] Central Processing Unit
	\item[HTML] HyperText Markup Language
	\item[IDE] Integrated Development Environment
	\item[JSON] JavaScript Object Notation
	\item[SQL] Structured Query Language
	
\end{siglas}
%% ===================================================================================

% The 'siglas' environment creates the list of acronyms.
\begin{siglas}
	% [TODO: Add your project's acronyms and abbreviations here.]
	% The format is: \item[ACRONYM] Full Name / Description
	
	\item[ABNT] Associação Brasileira de Normas Técnicas
	\item[API] Application Programming Interface
	\item[CPU] Central Processing Unit
	\item[HTML] HyperText Markup Language
	\item[IDE] Integrated Development Environment
	\item[JSON] JavaScript Object Notation
	\item[SQL] Structured Query Language
	
\end{siglas}
%% ===================================================================================

% The 'siglas' environment creates the list of acronyms.
\begin{siglas}
	% [TODO: Add your project's acronyms and abbreviations here.]
	% The format is: \item[ACRONYM] Full Name / Description
	
	\item[ABNT] Associação Brasileira de Normas Técnicas
	\item[API] Application Programming Interface
	\item[CPU] Central Processing Unit
	\item[HTML] HyperText Markup Language
	\item[IDE] Integrated Development Environment
	\item[JSON] JavaScript Object Notation
	\item[SQL] Structured Query Language
	
\end{siglas}

% --- List of Symbols (Optional) ---
% [TODO: Uncomment if you have a list of symbols]
\begin{simbolos}
   \item[$ \alpha $] -- Description of alpha symbol.
   \item[$ \beta $] -- Description of beta symbol.
\end{simbolos}
\cleardoublepage

% --- Table of Contents ---
\pdfbookmark[0]{\contentsname}{toc}
\tableofcontents*
\cleardoublepage

% ----------------------------------------------------------
% -- TEXTUAL ELEMENTS (CHAPTERS)
% ----------------------------------------------------------
\textual

% [TODO: Write your chapters in these separate .tex files]
% This modular approach is highly recommended for managing large documents.
\chapter{Introdução}\label{cap:intro}

\section{Contextualização}
A disseminação de conhecimento gratuito tem se mostrado uma prática cada vez mais relevante no contexto educacional, especialmente com o crescimento de plataformas online e de iniciativas de educação aberta. Professores, pesquisadores e outros produtores de conteúdo — especialmente nas ciências exatas, como ciência da computação, matemática e física — enfrentam, no entanto, diversas dificuldades ao tentar compartilhar materiais ricos, interativos e bem formatados de forma acessível e sem barreiras tecnológicas significativas. A produção e publicação de conteúdos como cursos e artigos frequentemente exige o domínio de múltiplas ferramentas, além de conhecimento técnico para lidar com formatações, códigos, fórmulas matemáticas e outros elementos fundamentais ao ensino moderno.

% \begin{listing}[h]
%     \lstinputlisting[language=c, firstline=10, lastline=20]{../../src/shell.c}
%     \caption{Trecho de código em c extraído do arquivo \texttt{/src/shell.c.}}
%     \label{list:label}
% \end{listing}


Apesar do avanço de editores e plataformas voltadas para a escrita digital, muitos desses ambientes não foram concebidos para atender às necessidades específicas de educadores que desejam compartilhar materiais com qualidade técnica e visual, mas sem abrir mão da simplicidade de uso. A ausência de funcionalidades integradas — como realce automático de código (code highlighting), suporte facilitado a fórmulas matemáticas escritas em LaTeX, criação de tabelas acessíveis e estruturação semântica do conteúdo — torna o processo de produção mais demorado e propenso a erros.

Nesse cenário, foi desenvolvido o FromCaio, um sistema web interativo voltado à criação e publicação de cursos e artigos educacionais. A proposta central é oferecer uma plataforma intuitiva que reduza as barreiras técnicas enfrentadas por educadores, promovendo a criação de conteúdo gratuito, acessível e com recursos avançados de formatação. O sistema permite que os próprios autores construam e publiquem seus materiais diretamente na plataforma, com o suporte de funcionalidades modernas que otimizam a experiência de escrita e de leitura.

Nesse cenário, foi desenvolvido o FromCaio, um sistema web interativo voltado à criação e publicação de cursos e artigos educacionais. A proposta central é oferecer uma plataforma intuitiva que reduza as barreiras técnicas enfrentadas por educadores, promovendo a criação de conteúdo gratuito, acessível e com recursos avançados de formatação. O sistema permite que os próprios autores construam e publiquem seus materiais diretamente na plataforma, contando com funcionalidades modernas que otimizam tanto a experiência de escrita quanto a de leitura.


% citações citacao1\cite{pressman2021} citacao2 \cite{KeRainerJr2019}.

\section{Motivação}
A educação tem o poder de transformar vidas. Essa afirmação, embora frequentemente repetida, se comprova de forma concreta na trajetória de inúmeros indivíduos — inclusive na do próprio autor deste trabalho. O ensino revelou-se um divisor de águas, não apenas por despertar o desejo de aprender e compartilhar conhecimento, mas também por dar sentido a aspectos da realidade que antes passavam despercebidos. Através da educação, temas antes invisíveis tornaram-se compreensíveis, e eventualmente até mesmo belos.

Aprender, apesar de desafiador, é uma das experiências mais enriquecedoras que alguém pode vivenciar. Seu impacto se manifesta de forma ampla: no plano intelectual, amplia a visão de mundo e desenvolve o pensamento crítico; no profissional, abre oportunidades e fortalece habilidades; e no pessoal, alimenta a curiosidade, o senso de propósito e a realização de contribuir com os outros.

Foi a partir dessa vivência que surgiu o desejo de colaborar ativamente para a ampliação do acesso ao conhecimento, valorizando especialmente os profissionais que se dedicam à produção científica e educacional. A motivação principal deste trabalho é criar um ambiente que respeite e reconheça esse esforço, oferecendo uma plataforma aberta, intuitiva e acessível tanto para quem ensina quanto para quem aprende.

A proposta do FromCaio nasce, portanto, da convicção de que democratizar o conhecimento vai além de simplesmente distribuí-lo. É preciso garantir sua qualidade, relevância e potencial transformador. O objetivo é proporcionar um espaço onde educadores possam produzir e compartilhar conteúdos que gerem impacto verdadeiro — tanto na vida de indivíduos quanto na atuação de empresas e instituições. Este projeto busca ser um passo concreto nessa direção.

\section{Objetivos}

\subsection{Objetivo Geral}

Desenvolver uma plataforma independente de publicação de materiais didáticos e científicos, com foco na autonomia dos autores, na permanência e portabilidade do conteúdo e no impacto profissional e social da produção intelectual.

\subsection{Objetivos Específicos}

\begin{itemize}
    \item Oferecer suporte a elementos essenciais para a produção técnico-científica, como trechos de código, expressões matemáticas e organização semântica do conteúdo;
    \item Viabilizar a publicação e a colaboração em projetos educacionais mesmo fora de contextos institucionais, eliminando barreiras técnicas e operacionais comuns em plataformas tradicionais;
    \item Adotar o formato \textit{markdown}, estendido com funcionalidades específicas da plataforma, a fim de garantir a longevidade, a portabilidade e o reaproveitamento dos materiais produzidos;
    \item Implementar recursos avançados como realce de sintaxe (\textit{code highlighting}) para múltiplas linguagens de programação;
    \item Desenvolver, em versões futuras, funcionalidades como a geração automática de gráficos interativos, ampliando as possibilidades pedagógicas e científicas;
    \item Estabelecer uma política de publicação responsável, com curadoria leve, baseada na expertise dos autores, a fim de assegurar a qualidade e a credibilidade dos conteúdos publicados.
\end{itemize}

\section{Justificativa}

A proposta deste trabalho se justifica por sua relevância em múltiplas esferas: acadêmica, educacional, social e tecnológica. Em um cenário em que o acesso ao conhecimento cresce continuamente, aumenta também a necessidade por ferramentas que não apenas permitam a publicação de conteúdo, mas que assegurem sua qualidade, acessibilidade e independência.

Apesar do avanço das plataformas digitais, ainda existem lacunas significativas — especialmente para educadores, pesquisadores e estudantes que desejam publicar ou colaborar na produção de conteúdo. Isso é ainda mais evidente entre aqueles que buscam alternativas gratuitas de publicação, sem depender de sistemas institucionais mais rígidos, os quais frequentemente impõem barreiras técnicas e operacionais. Essas limitações acabam desestimulando a criação de conteúdo educativo de qualidade — um processo que, por si só, já é desafiador — e que poderia ser significativamente fortalecido com a redução das barreiras técnicas envolvidas.


Ao contrário de ambientes virtuais de aprendizagem amplamente utilizados, como o Moodle — cuja proposta central é gerenciar cursos e atividades acadêmicas institucionais —, o \textit{FromCaio} foi concebido especificamente como uma plataforma de publicação autônoma de conteúdos didáticos e científicos. Embora o Moodle cumpra eficientemente seu papel em contextos formais de ensino, adaptá-lo por meio de extensões para atender a fluxos editoriais mais flexíveis exigiria modificações complexas e uma adesão a requisitos técnicos institucionais. O \textit{FromCaio}, por sua vez, busca justamente superar essas limitações, permite que professores e estudantes colaborem na criação de projetos, orientados institucionalmente ou não, sem que precisem se adequar a modelos rígidos de publicação, favorecendo a clareza, a didática e o controle autoral sobre o material produzido.

Essa independência é sustentada por decisões técnicas que favorecem tanto a longevidade quanto a portabilidade dos materiais produzidos. A adoção do formato \textit{markdown} — ainda que estendido com funcionalidades específicas da plataforma — permite a migração e o reaproveitamento de conteúdos de forma ágil e eficaz, inclusive a partir de outros formatos amplamente utilizados no meio acadêmico, como slides e roteiros de aula. Essa flexibilidade é ainda fortalecida pelo uso de ferramentas externas baseadas em inteligência artificial generativa, que auxiliam na conversão entre formatos distintos, preservando a coerência, a estrutura e a utilidade pedagógica do conteúdo ao longo de diferentes contextos de publicação.

Complementando esse conjunto de soluções, a plataforma oferece recursos que ampliam o potencial técnico e didático dos materiais. Um dos destaques é o suporte ao realce de sintaxe (\textit{code highlighting}) para mais de 40 linguagens de programação — incluindo C, Python, Assembly e Arduino —, número que continuará a crescer conforme novas necessidades surgirem e conteúdos forem adicionados. Versões futuras também preveem funcionalidades como a geração automática de gráficos interativos, abrindo novas possibilidades de aplicação pedagógica e científica.

Por fim, embora valorize a liberdade editorial, a proposta adota uma política de publicação responsável, pautada na credibilidade e na qualidade. A autoria é reservada a usuários com expertise comprovada — como professores universitários, profissionais especializados ou estudantes sob orientação adequada. Essa curadoria leve tem como objetivo assegurar que os conteúdos publicados estejam alinhados ao propósito da plataforma: promover impacto real e positivo na formação intelectual, profissional e pessoal dos leitores.

\section{Organização da Monografia}

Além deste capítulo, os demais estão organizados da seguinte forma:

\begin{itemize}
    \item Capítulo \ref{cap:ref_teorico}: Contém o referencial teórico, onde foram fundamentados os conceitos, técnicas e tecnologias utilizadas no decorrer do estudo. Além disso, também são apresentados os trabalhos relacionados;
    \item Capítulo \ref{cap:metodologia}: Neste capitulo é apresentado a metodologia, contendo as etapas realizadas no desenvolvimento do sistema;
    \item Capítulo \ref{cap:dev_sist}: São relatados as etapas da construção dos sistema, assim como os resultados alcançados no desenvolvimento e testes;
    \item Capítulo \ref{cap:conclusao}: Neste capitulo são apresentadas as conclusões dos autores junto as considerações finais e proposta de trabalhos futuros;
\end{itemize}
\chapter{Referencial Teórico}\label{cap:ref_teorico}

\section{Fundamentação}

Nesta seção são apresentados os conceitos teóricos e tecnologias adotadas que fundamentaram o desenvolvimento do SI proposto neste projeto. 

\subsection{Sistemas de Informação}

De modo geral, o desenvolvimento de software, incluindo o de SIs, pode ser descrito nas seguintes etapas \cite{pressman2021}:

\begin{enumerate}
    
    \item Concepção: compreende a identificação do domínio de aplicação do sistema, os perfis de usuários, o ambiente de execução, sua funcionalidade geral e limites de escopo;
   
    \item Elaboração: compreende a definição da base tecnologia do SI, da sua funcionalidade estrutural (arquitetura) e identificação dos requisitos/casos de uso;
   
    \item Construção: compreende na implementação e teste das estruturas que realizam a funcionalidade do sistema;
    
    \item Implantação: compreende na instalação do sistema no ambiente de operação, incluindo todos os ajustes e testes necessários.
    
\end{enumerate}

Ressalta-se que essas etapas estão apresentadas aqui de maneira genérica. Existem diversos modelos de processo de desenvolvimento de software que especificam e organizam essas atividades, cada um a sua maneira\cite{pressman2021}.


Os requisitos expressam as necessidades e restrições colocadas sobre o produto
de software que contribuem para a solução de algum problema do mundo real. Esta área
envolve elicitação, análise, especificação e validação dos requisitos de software\cite{SWEBOK}.

\subsection{Desenvolvimento web}

Com o crescimento na diversidade dos dispositivos tecnológicos, cada vez mais aplicações web tem se tornado a plataforma de escolha dos usuários finais. Isso se deve a flexibilidade de acesso, podendo ser acessado por qualquer dispositivo com acesso a navegadores de qualquer lugar do mundo \cite{ref6}. Desta forma diversos produtos de várias áreas optaram por migrar seus serviços para plataformas web. Por exemplo, aplicações de edição de texto, aplicativos de escritório e serviços, plataformas bancárias, sistemas de mercado de ações, entre outras.

Para que o desenvolvimento de uma aplicação web seja bem sucedido, é de suma importância aplicar conceitos provenientes da Engenharia de Software. Ginige e Marugesan (2001) \cite{ref7} sugerem os seguintes passos para atingir um desenvolvimento e implantação bem sucedido, passos esses que se influenciam e devem ser iterativos:

\begin{enumerate}
    \item Entender a funcionalidade geral do sistema, seu ambiente de operação, incluindo seus objetivos de negócios e requerimentos.
    \item Ter clareza sobre os \textit{Stakeholders}, ou seja, os principais usuários do sistema, a organização que precisa do sistema e quem é financia o desenvolvimento.
    \item Especificar os requisitos(iniciais) funcionais, técnicos e não técnicos. E reconhecer que esses podem não permanecer os mesmos.
    \item Desenvolver uma arquitetura que apresente os requirimentos técnicos e não técnicos.
    \item Identificar subprocessos para implementar na arquitetura. Caso esses subprocessos sejam muito complexos de gerenciar, quebra-los novamente até atingir tarefas de baixa complexidade de gerenciamento.
    \item Desenvolver e implementar os subprocessos.
    \item Incorporar mecanismos eficientes para gerenciar a evolução, mudanças e manutenção do sistema web.
    \item Identificar os problemas não técnicos.
    \item Medir a performance do sistema.
    \item Refinar e melhorar o sistema. 
    
\end{enumerate}

\subsection{Tecnologias e conceitos}

Nesta seção, são apresentadas as tecnologias utilizadas no desenvolvimento do sistema proposto, destacando suas características que justificam sua seleção para este trabalho.

\subsubsection{HTML e CSS}

HTML (HyperText Markup Language) e CSS (Cascading Style Sheets) são duas tecnologias fundamentais para a criação de páginas na web. O HTML é a linguagem de marcação responsável pela estruturação do conteúdo, definindo elementos como títulos, parágrafos, imagens, links e tabelas. Ele fornece a base sobre a qual todos os outros componentes de uma página são construídos, organizando as informações de forma hierárquica e semântica. Conforme observado em estudos, o HTML funciona como uma linguagem de marcação descritiva, onde os elementos são usados para descrever o significado semântico do conteúdo, ao invés de apenas formatá-lo \cite{bernard2005html}. Por outro lado, o CSS é a linguagem usada para controlar a apresentação visual, sendo descrito como uma ferramenta essencial para a separação de estrutura e design, permitindo uma personalização sofisticada e consistente da aparência de páginas web \cite{smith2008css}.

\subsubsection{Javascript e Typescript}

O JavaScript é uma das linguagens mais utilizadas na modernidade. Sua popularidade pode ser explicada pela sua alta usabilidade, além de sua natureza dinâmica e permissiva \cite{andreasen2017survey}. Por um lado, essa flexibilidade pode ser considerada favorável em termos de agilidade e facilidade no desenvolvimento, mas por outro, pode se tornar algo muito complexo em aplicações de escalas maiores. Com isso o Typescript surge para endereçar essa deficiência. Conhecido por ser um \textit{Superset} do Javascript, o Typescript enriquece o JS, com classes, interfaces, módulos e tipagem estática, assistindo os programadores de forma leve, flexível e de fácil usabilidade \cite{bierman2014understanding}.


\subsubsection{React e Next.js}

A complexidade e o dinamismo das aplicações modernas faz-se necessário a utilização de Frameworks e bibliotecas que auxiliam no desenvolvimento. React, biblioteca Javascript que pode estar definida na camada View de uma arquitetura MVC (Model, View, Controller) possibilita a criação de interfaces modulares, isto é feito através da construção de UIs a partir de pequenas unidades. Esses componentes poderão ser reutilizados e combinados em outros componentes que formarão a interface.\cite{react} 
O React permite a criação de Single Page Applications (SPA's), que são aplicações que carregam todos os recursos necessários na requisição inicial, e, em seguida, os componentes da página são substituídos por outros componentes a partir da interação do usuário.\cite{jadhav2015single}. SPA's contribuem para criação de aplicações modernas uma vez que, conteúdos dinamicamente carregados entregam uma experiencia para o usuário mais fluída, aproximando de aplicações Desktop. Mas por outro lado, esse tipo de construção não entrega somente aspectos positivos, SPA's são comumente conhecidos por apresentar uma alto trafego de API, além de complexidades em lidar com Search Engine Optimization (SEO)\cite{konshin2018next}.

Next.js é um Framework React, que somado a todos os benefícios já oferecidos pelo React, incorporam à experiencia de desenvolvimento recursos prontos para produção como Server Side Rendering (SSR), Static Site Generation (SSG) e Incremental Site Regeneration (ISR)\cite{dinku2022react}. Dessa forma, o Next.js enriquece o desenvolvimento a partir do React solucionando as deficiências previamente apresentadas.

\subsubsection{Nodejs}

Node.js é uma tecnologia que permite a execução de código JavaScript no lado do servidor, fora do ambiente do navegador. Antes de sua criação, o JavaScript era amplamente utilizado apenas para programação frontend, limitando seu uso ao lado do cliente. Com o surgimento do Node.js, em 2009, essa barreira foi quebrada, permitindo que desenvolvedores utilizassem JavaScript em todo o \textit{stack} de desenvolvimento, tanto no \textit{front-end} quanto no \textit{back-end}. A importância do Node.js na web é notável, pois ele oferece um ambiente de execução assíncrono e orientado à eventos, apresentando ótima capacidade de lidar com um grande número de conexões simultâneas com baixa latência e eficiência de recursos, Desta forma, o Node.js é ideal para aplicações de alto desempenho, como servidores web, APIs e sistemas de tempo real. Lee, J. and Kim, H., exaltam que a arquitetura baseada em eventos do Node.js permite uma escalabilidade horizontal significativa, reduzindo a sobrecarga do servidor e otimizando o uso de recursos \cite{lee2019nodejs}.

Em resumo, o Node.js transformou o JavaScript em uma linguagem de propósito geral, ampliando suas capacidades e solidificando sua posição como uma das tecnologias mais influentes e versáteis da web moderna.

\subsubsection{Express.js}

O Express.js é um framework minimalista e flexível para o desenvolvimento de aplicações web e APIs em Node.js. Ele simplifica a criação de servidores, oferecendo ferramentas e funcionalidades para gerenciar rotas, requisições, respostas, middleware e outras operações comuns no desenvolvimento backend. Por ser leve e extensível, o Express.js é amplamente utilizado, permitindo que os desenvolvedores criem desde aplicações simples até sistemas complexos e escaláveis. Ele também se integra facilmente com outros pacotes do ecossistema Node.js, tornando-o uma escolha popular para aplicações web modernas.

\subsubsection{Arquitetura Limpa}

A arquitetura limpa possibilita a criação de sistemas robustos, onde as regras de negócio são isoladas das interfaces externas, favorecendo a evolução contínua e a redução de acoplamento \cite{silva2018arquiteturalimpa}. Esse isolamento entre camadas contribui para um design de software que é mais sustentável e adaptável a mudanças futuras \cite{martins2017arquiteturaclean}. O conceito consiste em um design de software que organiza o código em camadas, cada uma com responsabilidades bem definidas, para tornar o sistema mais modular, flexível e fácil de manter. A ideia principal é separar a lógica de negócios (as regras principais da aplicação) da interface do usuário, dos detalhes técnicos, como bancos de dados e \textit{frameworks}. Na Arquitetura Limpa (Figura \ref{fig:CleanArq}), o sistema é dividido em quatro camadas principais:

    \begin{itemize}
        \item Entidades: Contém as regras de negócios mais gerais e não dependem de nada externo. São o núcleo do sistema.
        \item Casos de Uso: Define as regras de negócios específicas da aplicação, controlando o fluxo de dados entre as outras camadas.
        \item Interface de Adaptação: Serve como uma ponte entre os casos de uso e as camadas externas, como a interface do usuário ou banco de dados.
        \item Interface Externa: Inclui detalhes como \textit{frameworks}, bibliotecas, e interfaces de usuário. Essa camada pode ser trocada sem afetar o restante do sistema.
    \end{itemize}

Esta arquitetura ajuda a manter o código organizado e facilita a adaptação do sistema a novas tecnologias ou mudanças nos requisitos, pois cada camada é independente das outras. Isso significa que mudanças em uma camada não afetam as outras, resultando em um sistema mais robusto e com maior facilidade de evolução \cite{silva2018arquiteturalimpa}. Além disso, estudos indicam que o uso da arquitetura limpa aumenta significativamente a testabilidade do código, uma vez que as dependências entre módulos são minimizadas \cite{pereira2020testabilidadearquitetura}.




% \begin{figure}[h]
%     \centering
%     \includegraphics[width=300px]{imgs/clean-arch-2.png}
%     \caption{Arquitetura Limpa.}
%     \label{fig:CleanArq}
% \end{figure}


ciencias exatas, hospedagem, longevidade, portabilidade

interesse por áreas diversas, algo que deve continuar a mudar, vivencia pessoal com a educação

obj geral, autonomia dos autores, permanencia e portabilidade

obj especifico, oferecer suporte a elementos essenciais para produção tecnico-cientifica,
trecho de codigo, expressoes matematicas, organização e semantica do conteudo

lacunas significativas quando queremos nos referir a plataformas gratuitas, que não exijam vinculos

\section{Trabalhos Relacionados}

Esta Sessão apresenta os trabalhos relacionados que servem como base para este estudo. Serão discutidas abordagens, metodologias e resultados relevantes encontrados na literatura, destacando contribuições significativas e identificando limitações que motivam a proposta deste trabalho.

%  ideia geral, web ou nao, gratuito, 

O ClickUp \cite{clickup} é uma plataforma de gestão de projetos e produtividade amplamente reconhecida por sua aplicação em diversos setores, incluindo o ambiente educacional. A ferramenta disponibiliza recursos que possibilitam a organização, o acompanhamento e o gerenciamento de tarefas de maneira colaborativa, configurando-se como uma solução eficaz para a otimização de processos administrativos e acadêmicos em instituições de ensino superior. No entanto, sua versão gratuita é limitada ao uso individual, sendo necessária a assinatura de planos pagos para viabilizar a utilização colaborativa entre equipes e múltiplos usuários.

O artigo Sistema de Gerenciamento de Planos de Curso \cite{oliveirasistema} apresenta uma proposta de ferramenta voltada ao aprimoramento da gestão de processos em instituições de ensino superior, com ênfase no controle, monitoramento e execução das atividades relacionadas aos planos de curso. O sistema busca automatizar e padronizar os procedimentos de elaboração e gerenciamento desses planos, visando à otimização do tempo dos colaboradores e ao aumento da eficiência institucional. Contudo, uma limitação identificada no trabalho reside na ausência de uma interface do sistema, cuja implementação foi indicada como uma proposta para estudos futuros.

\chapter{Metodologia}\label{cap:metodologia}

Para alcançar o objetivo de desenvolver uma plataforma de publicação robusta, escalável e intuitiva, a construção do sistema FromCaio seguiu um processo de desenvolvimento iterativo e incremental. Esta abordagem permitiu a entrega de valor contínua e a adaptação a novos requisitos de forma flexível. O projeto foi estruturado nas seguintes macroetapas:

\begin{enumerate}
    \item Pesquisa e Definição das Tecnologias, cujos fundamentos são detalhados no Capítulo \ref{cap:ref_teorico}.
    \item Levantamento e Análise de Requisitos.
    \item Projeto da Arquitetura e Modelagem de Dados.
    \item Implementação e Validação do Sistema.
\end{enumerate}


\section{Levantamento e Análise de Requisitos}\label{section:req}

Nesta fase inicial, a coleta de requisitos foi realizada por meio de uma abordagem multifacetada. Primeiramente, conduziu-se uma análise de plataformas de publicação de conteúdo já consolidadas (como Medium, DEV.to e blogs acadêmicos) para identificar funcionalidades essenciais, padrões de interface e lacunas que o FromCaio poderia suprir. Em paralelo, foram realizadas discussões com o Professor Orientador para alinhar os objetivos do projeto com as necessidades de educadores e produtores de conteúdo técnico. A partir dessa análise, foram definidos os requisitos funcionais, como o sistema de autenticação, o editor baseado em Markdown, o suporte a LaTeX e o realce de sintaxe de código, e os requisitos não funcionais, como desempenho, usabilidade e portabilidade do conteúdo.


\section{Projeto da Arquitetura e Modelagem de Dados}

Com os requisitos definidos, o projeto focou na definição de uma arquitetura limpa e desacoplada, visando a manutenibilidade e a escalabilidade futuras da plataforma.

A arquitetura do sistema foi dividida em duas partes principais: o \textit{back-end} (servidor e API) e o \textit{front-end} (interface do usuário). Para o \textbf{back-end}, optou-se pelo uso de \textbf{Node.js} com o framework \textbf{Express.js} para a construção de uma API RESTful. Essa escolha se deve à performance suficiente em operações de entrada/saída (I/O) e ao vasto ecossistema de bibliotecas, que facilitam a implementação de funcionalidades como autenticação, processamento de Markdown e interação com o banco de dados. A lógica de negócio foi isolada em módulos específicos, garantindo que a camada de aplicação permanecesse independente do framework web.

Para a construção do \textbf{front-end}, foi escolhido o framework \textbf{Next.js}, que é baseado em \textbf{React}. A escolha do Next.js foi estratégica devido aos seus recursos de Renderização no Lado do Servidor (SSR) e Geração de Site Estático (SSG). Tais funcionalidades são cruciais para uma plataforma de conteúdo como o FromCaio, pois garantem um excelente desempenho de carregamento das páginas e otimização para motores de busca (SEO), tornando o conteúdo facilmente encontrável.

A \textbf{modelagem de dados} foi realizada com a criação de um diagrama entidade-relacionamento (DER) para estruturar as principais entidades do sistema, como `Usuários`, `Artigos`, `Cursos` e `Categorias`. O banco de dados foi projetado para garantir a integridade referencial e suportar de forma eficiente as consultas necessárias para a exibição dos conteúdos e gerenciamento da plataforma.

\section{Implementação e Validação do Sistema}

A fase de implementação seguiu uma abordagem iterativa, desenvolvendo e integrando as funcionalidades em ciclos. O desenvolvimento foi organizado da seguinte forma:

\begin{enumerate}
    \item \textbf{Desenvolvimento do Back-end (API):} Inicialmente, foi implementada a API com todos os \textit{endpoints} necessários para o CRUD (Create, Read, Update, Delete) das entidades, além das regras de negócio para permissões, publicação e processamento do conteúdo.
    \item \textbf{Desenvolvimento do Front-end:} Com a API funcional, a interface do usuário foi construída, consumindo os dados do \textit{back-end}. Cada componente da interface, como o editor de texto, a visualização de artigos e os painéis de usuário, foi desenvolvido de forma modular.
    \item \textbf{Validação:} Ao final de cada ciclo de implementação, foram realizados testes para validar as funcionalidades. Foram aplicados testes unitários nas regras de negócio críticas do \textit{back-end} e testes de integração para verificar a comunicação entre o \textit{front-end} e a API. Adicionalmente, foi realizada uma validação funcional manual, simulando o uso da plataforma por autores e leitores para garantir que os requisitos foram atendidos e que a experiência de uso era intuitiva e fluida.
\end{enumerate}

\section{Cronograma}

O cronograma a seguir apresenta as etapas planejadas para o desenvolvimento do projeto \textit{FromCaio}, abrangendo desde o planejamento inicial até a finalização e entrega da monografia.

\begin{table}[h!]
\centering
\caption{Cronograma de Desenvolvimento do Projeto FromCaio}
\label{tab:cronograma}
\renewcommand{\arraystretch}{1.3}
\begin{tabular}{@{}p{3.2cm} p{5.5cm} p{2.5cm} p{2.5cm}@{}}
\toprule
\textbf{Etapa} & \textbf{Atividade Específica} & \textbf{Início} & \textbf{Término} \\
\midrule

\multirow{3}{*}{\parbox[l]{3cm}{Planejamento\\e Arquitetura}}
& Detalhamento dos requisitos & 21/07/2025 & 27/07/2025 \\
& Definição da arquitetura e tecnologias & 21/07/2025 & 27/07/2025 \\
& Modelagem do banco de dados (DER) & 21/07/2025 & 27/07/2025 \\
\midrule

\multirow{4}{*}{\parbox[l]{3cm}{Desenvolvimento do Back-End}}
& Configuração do ambiente (API) & 28/07/2025 & 03/08/2025 \\
& Sistema de autenticação de usuários & 28/07/2025 & 03/08/2025 \\
& CRUD para Artigos e Cursos & 04/08/2025 & 10/08/2025 \\
& Lógica de processamento de conteúdo & 04/08/2025 & 10/08/2025 \\
\midrule

\multirow{4}{*}{\parbox[l]{3cm}{Desenvolvimento do Front-End}}
& Configuração do ambiente (UI) & 11/08/2025 & 17/08/2025 \\
& Páginas estáticas e painel do usuário & 11/08/2025 & 24/08/2025 \\
& Implementação do editor de texto & 18/08/2025 & 24/08/2025 \\
& Telas de visualização de conteúdo & 25/08/2025 & 31/08/2025 \\
\midrule

\multirow{3}{*}{\parbox[l]{3cm}{Integração e Testes}}
& Conexão do Front-end com a API & 25/08/2025 & 31/08/2025 \\
& Testes de ponta a ponta (E2E) & 01/09/2025 & 07/09/2025 \\
& Correção de bugs e refatoração & 01/09/2025 & 07/09/2025 \\
\midrule

\multirow{3}{*}{\parbox[l]{3cm}{Finalização e Documentação}}
& Validação final e ajustes & 08/09/2025 & 14/11/2025 \\
& Redação da monografia & 08/09/2025 & 14/11/2025 \\
& Preparação da apresentação & 08/09/2025 & 14/11/2025 \\
\bottomrule
\end{tabular}
\end{table}
\chapter{Desenvolvimento do Sistema}\label{cap:dev_sist}

\section{O sistema}

O sistema dispõe de uma funcionalidade que auxilia as coordenadorias e departamentos nas definição dos encargos semestrais. Para isso, os cursos devem ser cadastrados ao sistema e administradores vinculados aos mesmos. A partir disso, eles passarão a ser responsáveis pelo seu gerenciamento. Professores poderão ser cadastrados para que possam se dispor a ministrar disciplinas, informando dados sobre a mesma e sobre sua própria disponibilidade de horário. Por fim, o sistema auxiliará a administração das Coordenadorias nas definições de disciplinas e seus professores, avaliação dos planos de ensino, e por último, definir o quadro de horário do devido semestre.

\section{Requisitos}\label{section:dev:req}

\subsection{Atores}
O sistema dispõem de diferentes tipos de usuários que desempenham seus devidos papeis, sendo estes:

    \begin{description}
        \item[Administrador Geral:] Responsável pelo gerenciamento geral do sistema. É encarregado de criar os cursos e atribuir administradores aos mesmos. 
        \item[Administrador de Curso:] Qualificação usualmente atribuída aos Coordenadores, Vice Coordenadores e Secretários dos cursos, são responsáveis pelo gerenciamento específicos do seu curso, como informações básicas do mesmo, gestão dos usuários Professores, Membros do Colegiado e definição de horários.
        \item[Professor:] Papel atribuído aos professores dos cursos. Dentre outras responsabilidades, esses deverão se dispor a ministrar disciplinas, apresentando informações sobre as mesmas, como numero de vagas, plano de ensino e necessidade de laboratórios.
        \item[Membro do colegiado:] Usuários que serão responsáveis pela avaliação dos planos de ensino das disciplinas.
        \item[Chefe de Departamento:] Usuários que serão responsáveis pela gerenciamento dos Departamentos.
    \end{description}

\newpage
\subsection{Casos de uso}\label{subsec:casosdeuso}

lorem


\begin{figure}[h]
    \centering
    \includegraphics[width=450px]{imgs/caso_de_uso_2.png}
    \caption{Diagrama Casos de Uso.}
    \label{fig:Figura1}
\end{figure}


\begin{itemize}
    \item
        \begin{description}
               \item[Identificação] CDU001.
               \item[Nome] Cadastro de Usuário.
               \item[Ator Primário:] Todos os usuários.
               \item[Precondição:] Nenhuma.
               \item[Pós-condição:] Um retorno visual é apresentado e o usuário é redirecionado para tela de login.
               \item[Fluxo de Execução:] - 
                    \begin{enumerate}
                        \item O usuário informa seus dados.
                        \item O usuário seleciona a opção de cadastrar no sistema.
                    \end{enumerate}
        \end{description}
        
    \vspace{10pt}
        
    \item   
        \begin{description}
               \item[Identificação] CDU002.
               \item[Nome] Login.
               \item[Ator Primário:] Todos os usuários.
               \item[Precondição:] O usuário precisa estar cadastrado no sistema.
               \item[Pós-condição:] O usuário é redirecionado a tela principal do sistema.
               \item[Fluxo de Execução:] - 
                    \begin{enumerate}
                        \item O usuário insere suas credenciais.
                        \item O usuário seleciona a opção de entrar no sistema.
                    \end{enumerate}
        \end{description}

    \vspace{10pt}

    \item       
        \begin{description}
           \item[Identificação] CDU003.
           \item[Nome] Cadastrar Curso.
           \item[Ator Primário:] Administrador Geral.
           \item[Precondição:] O usuário deve estar logado no sistema. 
           \item[Pós-condição:] Um retorno visual é apresentado ao usuário e um novo curso é cadastrado no sistema.
           \item[Fluxo de Execução:] - 
                \begin{enumerate}
                    \item O usuário seleciona a opção de criar um novo curso.
                    \item O usuário deve inserir as informações necessárias para criação e confirmar ação.
                \end{enumerate}
        \end{description}

     \vspace{10pt}
    
    \item 
        \begin{description}
            \item[Identificação] CDU004.
           \item[Nome] Cadastrar Período.
           \item[Ator Primário:] Administrador Geral.
           \item[Precondição:] O usuário deve estar logado no sistema. 
           \item[Pós-condição:] Um retorno visual é apresentado ao usuário e um novo período é cadastrado no sistema.
           \item[Fluxo de Execução:] - 
                \begin{enumerate}
                    \item O usuário seleciona a opção de criar um novo período.
                    \item O usuário deve inserir as informações necessárias para criação e confirmar ação.
                \end{enumerate}
        \end{description}

    \vspace{10pt}

    \item 
        \begin{description}
            \item[Identificação] CDU005.
            \item[Nome] Cadastrar Departamento.
            \item[Ator Primário:] Administrador Geral.
            \item[Precondição:] O usuário deve estar logado no sistema. 
            \item[Pós-condição:] Um retorno visual é apresentado ao usuário e um novo departamento é cadastrado no sistema.
            \item[Fluxo de Execução:] -
                \begin{enumerate}
                    \item O usuário seleciona a opção de criar um novo departamento.
                    \item O usuário deve inserir as informações necessárias para criação e confirmar ação.
                \end{enumerate}
        \end{description}

    \item 
        \begin{description}
            \item[Identificação] CDU006.
            \item[Nome] Selecionar período.
            \item[Ator Primário:] Todos os usuários.
            \item[Precondição:] O usuário deve estar logado no sistema. 
            \item[Pós-condição:] As telas se adéquam ao período selecionado.
            \item[Fluxo de Execução:] -
                \begin{enumerate}
                    \item O usuário seleciona o período desejado a partir da lista.
                \end{enumerate}
        \end{description}

    \vspace{10pt}
    
    \item 
        \begin{description}
            \item[Identificação] CDU007.
            \item[Nome] Vincular professores ao departamento.
            \item[Ator Primário:] Administrador de Departamento.
            \item[Precondição:] O usuário deve estar logado no sistema e estar vinculado a um curso como administrador. 
            \item[Pós-condição:] Um retorno visual é apresentado ao usuário e um ou mais professores serão vinculados ao departamento.
            \item[Fluxo de Execução:] -
                \begin{enumerate}
                    \item O usuário deve selecionar a opção de cadastrar professores ao departamento.
                    \item O usuário deve inserir o nome do professor e selecionar a partir da lista.
                    \item O usuário confirma a ação e um retorno visual é apresentado.
                \end{enumerate}
        \end{description} 

    \vspace{10pt}

    \item 
        \begin{description}
           \item[Identificação] CDU008.
           \item[Nome] Vincular professor a uma oferta de disciplina.
           \item[Ator Primário:] Administrador de Departamento.
           \item[Precondição:] Deverá existir ofertas de disciplinas cadastradas e pendentes no sistema. 
           \item[Pós-condição:] Um retorno visual é apresentado ao usuário, o professor selecionado será notificado da ação.
           \item[Fluxo de Execução:] -
                \begin{enumerate}
                    \item O usuário deve selecionar a disciplina para confirmar o professor.
                    \item O usuário deve selecionar qual o professor escolhido para ministrar a disciplina.
                \end{enumerate}
        \end{description} 
    \vspace{10pt}

     \item 
        \begin{description}
           \item[Identificação] CDU009.
           \item[Nome] Finalizar edição de ofertas de disciplinas.
           \item[Ator Primário:] Administrador de Departamento.
           \item[Precondição:] Deverá existir ofertas de disciplinas cadastradas no sistema. 
           \item[Pós-condição:] Um retorno visual é apresentado ao usuário, e a edição de ofertas é bloqueada.
           \item[Fluxo de Execução:] -
                \begin{enumerate}
                    \item O usuário deve navegar para a tela de ofertas.
                    \item O usuário deve selecionar a opção de bloquear a edição de ofertas.
                \end{enumerate}
        \end{description} 
    \vspace{10pt}

    \item 
        \begin{description}
           \item[Identificação] CDU010.
           \item[Nome] Enviar lista de professores para ofertas de disciplina.
           \item[Ator Primário:] Administrador de Departamento.
           \item[Precondição:] Deverá existir ofertas de disciplinas cadastradas no sistema. 
           \item[Pós-condição:] Um texto informando os professores de cada oferta é disponibilizado ao usuário.
           \item[Fluxo de Execução:] -
                \begin{enumerate}
                    \item O usuário deve navegar para a tela de ofertas.
                    \item O usuário deve selecionar a opção de exibir texto.
                \end{enumerate}
        \end{description} 
    \vspace{10pt}

    \item 
        \begin{description}
           \item[Identificação] CDU011.
           \item[Nome] Avaliar plano de Ensino.
           \item[Ator Primário:] Membro do Colegiado.
           \item[Precondição:] Deverá existir planos de ensino cadastrado às ofertas no sistema. 
           \item[Pós-condição:] Um retorno visual é apresentado ao usuário, e o plano de ensino é marcado como aprovado ou rejeitado.
           \item[Fluxo de Execução:] -
                \begin{enumerate}
                    \item O usuário deve selecionar o plano de ensino a ser avaliado.
                    \item O usuário deve selecionar a opção de aprovação e rejeição.
                    \begin{enumerate}
                    \item Caso seja selecionado a opção de rejeição, uma justificativa deve ser apresentada.
                    \end{enumerate}
                \end{enumerate}
        \end{description} 
    \vspace{10pt}

    \item 
        \begin{description}
           \item[Identificação] CDU012.
           \item[Nome] Cadastrar disciplina.
           \item[Ator Primário:] Administrador do Curso.
           \item[Precondição:] Deverá existir um curso cadastrado no sistema.
           \item[Pós-condição:] Um retorno visual é apresentado ao usuário, e uma nova disciplina é vinculada ao curso.
           \item[Fluxo de Execução:] -
                \begin{enumerate}
                    \item O usuário deve selecionar o curso cuja disciplina será criada.
                    \item O usuário informa todas as informações necessárias e clica em adicionar.
                \end{enumerate}
        \end{description} 
    \vspace{10pt}

    \item 
        \begin{description}
           \item[Identificação] CDU013.
           \item[Nome] Criar oferta de disciplina.
           \item[Ator Primário:] Administrador do Curso.
           \item[Precondição:] Deverá existir um curso cadastrado no sistema com disciplinas vinculadas ao mesmo. 
           \item[Pós-condição:] Um retorno visual é apresentado ao usuário, e novas ofertas são vinculadas ao curso e departamento.
           \item[Fluxo de Execução:] -
                \begin{enumerate}
                    \item O usuário deve selecionar a opção de ofertar disciplinas.
                    \begin{enumerate}
                    \item Se o usuário selecionar a opção de ofertar uma disciplina, a turma deve ser inserida.
                    \item Se o usuário selecionar a opção de ofertar disciplinas em pacote, o usuário deverá selecionar se ofertará as disciplinas obrigatórias, opcionais, ou ambas.
                \end{enumerate}
                \end{enumerate}
        \end{description} 
    \vspace{10pt}

    \item 
        \begin{description}
           \item[Identificação] CDU014.
           \item[Nome] Adicionar membros do colegiado.
           \item[Ator Primário:] Administrador do Curso.
           \item[Precondição:] Deverá existir um curso cadastrado no sistema, assim como usuários.
           \item[Pós-condição:] Um retorno visual é apresentado ao usuário, e o usuário selecionado é vinculado ao curso como membro do colegiado.
           \item[Fluxo de Execução:] -
                \begin{enumerate}
                    \item O usuário deve selecionar a opção de adicionar membros ao colegiado.
                    \item O usuário deve selecionar a pessoa escolhida na lista apresentada.
                    \item O usuário deve confirmar a ação.
                \end{enumerate}
        \end{description} 
    \vspace{10pt}


    \item 
        \begin{description}
           \item[Identificação] CDU015.
           \item[Nome] Solicitar professor para oferta de disciplina.
           \item[Ator Primário:] Administrador do Curso.
           \item[Precondição:] Deverá existir ofertas de disciplinas cadastradas no sistema. 
           \item[Pós-condição:] Um texto solicitando os professores para cada oferta é disponibilizado ao usuário.
           \item[Fluxo de Execução:] -
                \begin{enumerate}
                    \item O usuário deve navegar para a tela de ofertas.
                    \item O usuário deve selecionar a opção de exibir texto.
                \end{enumerate}
        \end{description} 
    \vspace{10pt}

    \item 
        \begin{description}
           \item[Identificação] CDU016.
           \item[Nome] Cadastrar plano de ensino a oferta.
           \item[Ator Primário:] Professor.
           \item[Precondição:] Deverá existir ofertas de cadastradas no sistema e o professor vinculado a mesma. 
           \item[Pós-condição:] O plano de ensino é vinculado a oferta.
           \item[Fluxo de Execução:] -
                \begin{enumerate}
                    \item O usuário deve escolher a disciplina, a ele atribuído, que deseja inserir o plano.
                    \item O usuário deve inserir as informações necessárias.
                    \item O usuário deve confirmar a ação.
                \end{enumerate}
        \end{description} 
    \vspace{10pt}




\end{itemize}



\section{Modelagem do Banco de Dados}\label{section:dev:model}

De acordo com as definições dos requisitos realizado na Sessão \ref{section:dev:req}, foi elaborado a elucidação das entidades que compõem o projeto, assim como suas relações. Dessa forma, a modelagem do banco de dados do sistema segue a estrutura apresentada na Figura \ref{fig:Figura2}.

\begin{figure}[h]
    \centering
    \includegraphics[width=350px]{imgs/model_db.png}
    \caption{Modelagem Banco de Dados.}
    \label{fig:Figura2}
\end{figure}

Foi escolhido o PostgreSQL como sistema gerenciador de banco de dados. Sua escolha se deu por ser um SGBD robusto e de código aberto. A construção foi feita com o auxilio do TypeORM, biblioteca que assiste nas operações junto ao banco de dados, abstraindo complexidades e trabalhando de forma orientada a objetos.


\section{Arquitetura do Sistema}

Conforme as necessidades identificadas, o sistema foi estruturado nas seguintes componentes: uma camada de frontend, que é voltada para o usuário final e tem como função exibir a interface gráfica, além de permitir a interação direta com o usuário; uma camada de backend, responsável pelo processamento das requisições recebidas do frontend, pela interação com o banco de dados, pela execução da lógica de negócios, pela validação e manipulação de dados, bem como pelo envio de respostas ao frontend; e, por fim, uma camada de banco de dados, onde os dados da aplicação são armazenados, recuperados e manipulados. Essa arquitetura é representada na Figura \ref{fig:DigImpl}.

\begin{figure}[h]
    \centering
    \includegraphics[scale=0.8]{imgs/diagrama_implantacao.png}
    \caption{Diagrama de Implantação.}
    \label{fig:DigImpl}
\end{figure}



Dentre as diversas bibliotecas e implementações presentes no sistema, destacam-se aquelas que desempenham um papel fundamental em sua caracterização e funcionamento. O cliente utiliza Next.js\footnote{https://nextjs.org}, com o suporte do Ant Design\footnote{https://ant.design}  e Styled Components\footnote{https://styled-components.com} para a criação e estilização das interfaces de usuário. A comunicação entre o frontend e o backend é realizada através do Axios\footnote{https://axios-http.com/ptbr/docs/intro}, responsável por efetuar as requisições HTTP. O backend, por sua vez, foi desenvolvido utilizando Express.js\footnote{https://expressjs.com/pt-br} e faz uso do TypeORM\footnote{https://typeorm.io/} para o gerenciamento das consultas ao banco de dados.


O fluxo de comunicação entre essas camadas pode ser descrito da seguinte forma:

\begin{enumerate}
    \item Requisição do Usuário: O usuário interage com a interface do frontend (por exemplo, clicando em um botão de "Salvar").
    \item Requisição HTTP: O frontend captura essa interação e envia uma requisição HTTP ao backend, especificando o tipo de ação a ser realizada (como salvar um novo registro no banco de dados).
    \item Processamento no Backend: O backend recebe a requisição, valida os dados recebidos, e se necessário, realiza operações de leitura ou escrita no banco de dados.
    \item Interação com o Banco de Dados: O backend se comunica com o banco de dados para armazenar, recuperar ou manipular dados conforme solicitado.
    \item Resposta ao Frontend: Após processar a requisição, o backend envia uma resposta de volta ao frontend. Essa resposta pode incluir dados (como uma lista de registros) ou uma confirmação de que a operação foi bem-sucedida.
    \item Atualização da Interface: O frontend recebe a resposta e, com base nos dados retornados, atualiza a interface do usuário, exibindo novas informações ou notificações conforme necessário.
\end{enumerate}

Para o lançamento da aplicação foi utilizado a Vercel\footnote{https://vercel.com}, que fornece ferramentas de desenvolvedor e infraestrutura em nuvem para construir, dimensionar e proteger serviços web. Nessa etapa foi utilizado os serviços em nuvem da plataforma, onde foi disponibilizado os componentes do sistema para a internet.


% diagrama de implantação
% explicando comunicação
% comunicação
% cada parte
% componentes do sistema
% deploy
% serviço de banco
% fluxo de dados
% sermaias abstrado citar sem necessariamente sem mostrar codigo





\section{Construção do Sistema}

Nesta sessão, será detalhado como cada componente discutido neste capítulo contribui para a construção do sistema. Nas Sessões \ref{subsec:constr:admin} a \ref{subsec:constr:professor}, são apresentadas as telas e os código dos casos de uso do sistema.


\subsection{Funcionalidades Geral e Administrativa} \label{subsec:constr:admin}

Nesta Sessão, serão apresentadas as telas comuns a todos os usuários do sistema e também as do usuário administrador.

\subsubsection{Cadastro de Usuário e Login}
Ao acessar o sistema, a primeira página apresentada ao usuário é a de Login (CDU01), seguida pela de Registro (CDU2). Nestes primeiros componentes do sistema, o que se destaca são a criação do modelo de usuário (Listagem \ref{list:UserModel}), representando a estrutura do dado presente no banco de dados em notação orientada a objeto, o mecanismo utilizando para criação de usuários, armazenamento de senhas e login no sistema, além da própria interface.


\begin{figure}[H]
    \centering
    \begin{subfigure}[h]{0.49\textwidth}
        \centering
        \includegraphics[scale=0.14]{imgs/login.png}
        \caption{Tela de Login}
        \label{fig:card}
    \end{subfigure}
    \begin{subfigure}[h]{0.49\textwidth}
        \centering
        \includegraphics[scale=0.14]{imgs/registro.png}
        \caption{Tela de registro}
        \label{fig:ccard1}
    \end{subfigure}
     \caption{Telas de Login e Registro}
    \label{fig:criacao:curso:dpto}
\end{figure}

% \begin{figure}[h]
%     \centering
%     \includegraphics[scale=0.3]{imgs/login.png}
%     \caption{Tela de Login.}
%     \label{fig:Login}
% \end{figure}

% \begin{figure}[h]
%     \centering
%     \includegraphics[scale=0.3]{imgs/registro.png}
%     \caption{Tela de Registro.}
%     \label{fig:Registro}
% \end{figure}


Na Listagem {\ref{list:UserModel}} apresentada, é possível observar a utilização do TypeORM, onde são descritas cada uma das propriedades do modelo de usuário e suas respectivas relações. A classe resultante possibilita a manipulação dos usuários no código, e, por meio das funcionalidades fornecidas pela biblioteca, permite a interação com o banco de dados.

\begin{listing}[h]
    \begin{minted}[linenos, bgcolor=lightgray, fontsize=\footnotesize]{typescript}
@Entity("user")
class User {
  @PrimaryGeneratedColumn("uuid")
  id!: string;

  @IsNotEmpty()
  @IsString()
  @Column({
    length: 128,
  })
  name!: string;
  
  @IsNotEmpty()
  @IsString()
  @Column({
    length: 256,
    select: false,
  })
  password!: string;
}
    \end{minted}
    \caption {Modelo de Usuário.}
    \label{list:UserModel}
\end{listing}

Para o registro do usuário, destacamos as principais validações cujas são a unicidade do e-mail fornecido \ref{list:registroCDU}, esta caso falhe apresenta retorna um erro ao usuário e a criptografia baseada em \textit{Hash} para armazenamento seguro da senha dos usuários.

\newpage

\begin{listing}[h]
    \begin{minted}[linenos, bgcolor=lightgray]{typescript}
const user = await this.repository.findOne(email, "email");

if (user) {
  Logger.error("validation failed.");
  throw new DuplicatedEntityError("Email already in use!");
}

const newUserData = {
  name,
  email,
  password: bcrypt.hashSync(password, 10),
};
    \end{minted}
    \caption{Registro de Usuário.}
    \label{list:registroCDU}
\end{listing}


lorem

\begin{listing}[h]
    \begin{minted}[linenos, bgcolor=lightgray]{typescript}
const user = (await this.repository.findOne(email, "email")) as User;

if (!user) {
  const message = "User not found";
  Logger.info(message);
  throw new AuthenticateFailError(message);
}

const { password: userPassword } = 
    (await this.repository.getUserPassword(user.id)) as User;

const userRoles = this._getUserRoles(user);

if (await bcrypt.compare(password, userPassword)) {
  const token = jwt.sign(
    { id: user.id, userRoles: userRoles },
    process.env.JWT_TOKEN as string,
    {
      expiresIn: process.env.JWT_EXPIRE,
    }
  );

  const resData = {
    id: user.id,
    name: user.name,
    email: user.email,
    roles: userRoles,
    token,
  };

  return resData;
}
    \end{minted}
    \caption{Login.}
    \label{list:login}
\end{listing}


\subsubsection{Painel Geral e Minha Conta}

O Painel Geral da aplicação é apresentado conforme a Figura \ref{fig:dashboard:mark}. É possível visualizar todos os módulos no menu principal localizado no topo da tela, os quais serão abordados em detalhes nas seções correspondentes a cada funcionalidade. Nesta etapa, destacamos a estrutura da interface da aplicação, ressaltando o uso do \textit{framework} React, no qual cada pequena estrutura é denominada "Componente". Esses componentes são reutilizados ao longo das diferentes páginas, variando apenas os dados exibidos. De forma geral a aplicação é estruturada com um componente \textit{Header}, destacado em vermelho na Figura \ref{fig:dashboard:mark}, ao qual agrupa os módulos e fornece o roteamento entre os mesmos, reúne funcionalidades que necessitam fácil manuseio como sair do sistema e a troca de período. Um \textit{Sider}, em amarelo, que apresenta listagens e navegabilidade em funcionalidades especificas ao módulo, e por fim um \textit{Content}, representado na figura em azul, este responsável por apresentar as informações e ações do sistema.

Minha Conta é a primeira página apresentada ao usuário ao fazer o login no sistema. Neste módulo possibilita-se a manutenção de conta permitindo funcionalidades como troca de senha e e-mail. Para tal função é utilizado um formulário para coletar as informações fornecidas pelo usuário (Listagem \ref{list:email}).

\begin{listing}[!h]
    \begin{minted}[linenos, bgcolor=lightgray]{java}
 <PageContent>
      <SForm
        form={form}
        name="basic"
        layout="vertical"
        initialValues={{ remember: true }}
        onFinish={handleSubmit}
        onFinishFailed={onFinishFailed}
        autoComplete="off"
      >
        <Form.Item
          label="Senha"
          name="password"
          rules={[
            {
              required: true,
              message: "Por favor insira sua senha atual!",
            },
          ]}
        >
          <Input.Password placeholder="Insira sua senha atual" />
        </Form.Item>
        <Form.Item
          label="Novo Email"
          labelAlign="left"
          name="email"
          rules={[
            {
              required: true,
              type: "email",
              message: "Insira um email válido",
            },
          ]}
        >
          <Input placeholder="Insira seu novo email" />
        </Form.Item>
        <Form.Item>
          <div style={{ display: "flex" }}>
            <Button type="primary" htmlType="submit">
              Confirmar
            </Button>
          </div>
        </Form.Item>
      </SForm>
</PageContent>
    \end{minted}
    \caption{Formulário troca de e-mail.}
    \label{list:email}
\end{listing}



\begin{figure}[h]
    \centering
    \includegraphics[scale=0.23]{imgs/pagina-minha-conta.png}
    \caption{Minha Conta e principais Componentes da interface.}
    \label{fig:dashboard:mark}
\end{figure}



\subsubsection{Pagina de Período}\label{subsec:pag-periodo}

Nesta página (\ref{fig:periodo}), o administrador tem a capacidade de criar períodos que serão utilizados na aplicação. No \textit{Sider}, são listados os períodos já criados, os quais podem ser selecionados para exibição de suas informações no \textit{Content}. Destaca-se nesta página o uso de listagem infinita, apresentado na Listagem \ref{list:listagem-infinita} que realizará a paginação automática à medida que o usuário navega pelos períodos criados. O código dessa listagem apresenta o componente React responsável por esse comportamento, onde podemos verificar as propriedades fornecidas ao componente que indicam o funcionamento da listagem infinita, onde tem-se a função \textit{LoadMoreData} responsável pela busca de novos itens e a propriedade \textit{hasMore} como variável de controle sob a necessidade da busca dos mesmos.


\begin{listing}[h]
    \begin{minted}[linenos, bgcolor=lightgray]{bash}
<InfiniteScroll
          dataLength={data.length}
          next={loadMoreData}
          hasMore={data.length < total}
          loader={<Skeleton avatar paragraph={{ rows: 1 }} active />}
          endMessage={<Divider plain>Não há mais itens! </Divider>}
          scrollableTarget="scrollableDiv"
>
    <AlternateList
        itemLayout="vertical"
        dataSource={data}
        locale={{
          emptyText: (
            <Empty
              image={Empty.PRESENTED_IMAGE_SIMPLE}
              description={false}
            />
          ),
        }}
        renderItem={(item: IPeriod) => (
          <List.Item>
            <Link href={`/dashboard/periods/${item.id}`}>
              <Card
                className={
                  Number(pId) === item.id ? "selected" : "notSelected"
                }
                hoverable
              >{`Periodo: ${item.code}`}</Card>
            </Link>
          </List.Item>
        )}
      />
</InfiniteScroll>
    \end{minted}
    \caption{Código listagem infinita.}
    \label{list:listagem-infinita}
\end{listing}


\begin{figure}[h]
    \centering
    \includegraphics[scale=0.23]{imgs/periodo.png}
    \caption{Página Período}
    \label{fig:periodo}
\end{figure}



\newpage

\subsubsection{Paginas de Curso e de Departamento}

Nessas páginas o sistema permite a criação e manutenção dos Departamentos e Cursos. Para o usuário administrador, responsável pela criação dos mesmos, é apresentado a opção de criação, demonstrada na Figura \ref{fig:tela:curso}, acima da listagem dos respectivos registros que seguem a mesma logica apresentada na Sessão \ref{subsec:pag-periodo}. Ao selecionar a opção será disponibilizado ao usuário os devidos formulários (Figura \ref{fig:criacao:curso:dpto}), estes deverão ser preenchidos com as informações necessárias para criação de cada registro. A Listagem \ref{list:insert-curso} apresenta o código relacionado a inserção dos dados no banco de dados, onde é possível visualizar como as operações de inserção são realizadas pelo TypeORM, sendo realizada a partir de métodos do repositório fornecido pela biblioteca.

\begin{figure}[h]
    \centering
    \includegraphics[scale=0.23]{imgs/tela-curso.png}
    \caption{Página Curso}
    \label{fig:tela:curso}
\end{figure}


\begin{figure}[h]
    \centering
    \begin{subfigure}[h]{0.4\textwidth}
        \centering
        \includegraphics[scale=0.23]{imgs/criar-curso.png}
        \caption{Formulário criação de Curso.}
        \label{fig:card}
    \end{subfigure}
    \begin{subfigure}[h]{0.4\textwidth}
        \centering
        \includegraphics[scale=0.23]{imgs/criar-departemento.png}
        \caption{Formulário criação de Departamento}
        \label{fig:ccard1}
    \end{subfigure}
     \caption{Formulários de criação de curso e departamento}
    \label{fig:criacao:curso:dpto}
\end{figure}


\begin{listing}[h]
    \begin{minted}[linenos, bgcolor=lightgray]{bash}
async save(data: Course) {
    const repository = getRepository(Course);

    if (data.admins && data.admins.length) {
      data.admins.forEach((admin) => {
        admin.createdAt = new Date();
      });
    }

    return repository.save(data);
}
    \end{minted}
    \caption{Código inserção de curso.}
    \label{list:insert-curso}
\end{listing}



\subsection{Funcionalidades do Curso} \label{subsec:constr:curso}


Após a criação de um curso pelo Administrador, o usuário designado como Coordenador do Curso receberá automaticamente o papel de Administrador de Curso no sistema, com base no vínculo estabelecido. Dessa forma, o Coordenador assumirá a responsabilidade de gerenciar os cursos a ele associados. Nesta Sessão, serão apresentadas as funcionalidades disponíveis nesse módulo.

Ao selecionar um curso, as respectivas informações serão exibidas no \textit{Content} da página, conforme ilustrado na Figura \ref{fig:tela:curso}. Logo acima dessas informações, encontra-se um menu de navegação que disponibiliza todas as funcionalidades relacionadas ao curso, tais como Disciplina, Colegiado, Oferta e Avaliações de Plano. O código apresentado na Listagem \ref{list:nav:curso} demostra como os itens da navegação são gerenciados.


\begin{listing}[h]
    \begin{minted}[linenos, bgcolor=lightgray]{bash}
function CoursesPage(props: CoursePageProps) {
  const { course } = props;

  const items = [
    {
      label: (
        <span>
          <InfoCircleOutlined /> Informações
        </span>
      ),
      key: "item-1",
      children: <CourseInfo courseInfo={course} />,
    },
  ];

  return (
    <PageContent>
      <Tabs
        defaultActiveKey="item-1"
        centered
        style={{ width: "100%" }}
        items={items}
      />
    </PageContent>
  );
}
    \end{minted}
    \caption{Código navegação conteúdo do Curso.}
    \label{list:nav:curso}
\end{listing}


\subsubsection{Disciplinas}

Essa página (\ref{fig:disciplina:curso}) permite a visualização e o gerenciamento das disciplinas do curso selecionado, utilizando um componente denominado \textit{Collapse}. Esse componente exibe uma lista de painéis inicialmente recolhidos, possibilitando sua expansão para a apresentação completa dos registros correspondentes.


\begin{figure}[h]
    \centering
    \includegraphics[scale=0.27]{imgs/disciplina-curso.png}
    \caption{Disciplinas do Curso}
    \label{fig:disciplina:curso}
\end{figure}

\subsubsection{Colegiado}

Com o objetivo de vincular usuários ao colegiado, um formulário é exibido acima da lista de usuários já cadastrados, conforme demonstrado na Figura \ref{fig:colegiado:curso}. Conforme exemplificado na Sessão \ref{subsec:constr:curso}, uma vez vinculado a um curso, o usuário recebe o nível de acesso de Membro do Colegiado na aplicação, sendo concedidos os acessos correspondentes.


\begin{figure}[h]
    \centering
    \includegraphics[scale=0.25]{imgs/colegiado-curso.png}
    \caption{Membros do Colegiado do Curso}
    \label{fig:colegiado:curso}
\end{figure}



\subsubsection{Ofertas do curso} \label{subsec:oferta:curso}

A página Ofertas do Curso (Figura \ref{fig:oferta:curso}) é projetada para o gerenciamento das ofertas de disciplinas vinculadas a um curso, que serão submetidas ao Departamento responsável. O usuário pode optar por ofertar todas as disciplinas do curso de forma simultânea (Figura \ref{fig:oferta:curso1}), com a possibilidade de definir se deseja ofertar apenas as disciplinas obrigatórias, apenas as opcionais, ou ambas. Esse recurso permite a criação de ofertas em larga escala, otimizando o tempo e garantindo a disponibilidade das disciplinas conforme as necessidades curriculares. Ao utilizar essa funcionalidade, o Sistema irá pressupor que as disciplinas serão de turma única, que poderá ser alterado posteriormente. Como alternativa, a página oferece a opção de ofertar disciplinas individualmente (Figura \ref{fig:oferta:curso2}), permitindo que o usuário especifique o numero de vagas e o nome da turma para cada disciplina, possibilitando a criação de mais de uma turma de uma disciplina por semestre. Após a criação de uma oferta, esta estará imediatamente disponível ao Departamento para que seja realizado o cadastro dos professores responsáveis pela ministração da disciplina. Para fins de formalização, a página dispõe de uma funcionalidade que permite a geração automática de um texto de solicitação ao Departamento, contendo todas as ofertas criadas e requisitando oficialmente a disponibilização das disciplinas correspondentes, conforme apresentado na Figura \ref{fig:text:dpto}. Além disso, a interface apresenta uma tabela com todas as ofertas cadastradas, exibindo informações como o nome da disciplina, o tipo (obrigatória ou opcional), numero de vagas, semestre, o nome da turma e os professores vinculados após o processo do Departamento.


\begin{figure}[h]
    \centering
    \includegraphics[scale=0.25]{imgs/ofertas-curso.png}
    \caption{Ofertas do Curso}
    \label{fig:oferta:curso}
\end{figure}


\begin{figure}[h]
    \centering
    \includegraphics[scale=0.25]{imgs/texto-curso.png}
    \caption{Texto gerado para envio ao departamento}
    \label{fig:text:dpto}
\end{figure}


\begin{figure}[h]
    \centering
     \begin{subfigure}[h]{0.4\textwidth}
        \centering
        \includegraphics[scale=0.30]{imgs/adc-oferta-bulk.png}
        \caption{Formulário criação de oferta em massa}
        \label{fig:oferta:curso1}
    \end{subfigure}
    \begin{subfigure}[h]{0.4\textwidth}
        \centering
        \includegraphics[scale=0.30]{imgs/adc-oferta-unica.png}
        \caption{Formulário criação de oferta única}
        \label{fig:oferta:curso2}
    \end{subfigure}
    \caption{Formulários de ofertas}
    \label{fig:oferta:curso-group}
\end{figure}


\begin{listing}[h]
    \begin{minted}[linenos, bgcolor=lightgray]{bash}
const alreadyExistingSubjects =
      await this.subjectOfferRepository.findExistentBulkSubject(
        courseId,
        periodId,
        addRequired,
        addOptional
      );

    const alreadyExistingSubjectsIDs = alreadyExistingSubjects.map(
      (e) => e.subject.id
    );

    const nonExistentSubject = courseSubjects.filter(
      (e) => !alreadyExistingSubjectsIDs.includes(e.id)
    );

    const requiredSubjectOffersData = nonExistentSubject
        .map((eachSubject) => {
          return Object.assign(new SubjectOffer(), {
            period: {
              id: periodId,
            },
            subject: {
              id: eachSubject.id,
            },
            places: eachSubject.places,
      });
});
    \end{minted}
    \caption{Código criação de ofertas em pacote}
    \label{list:bulk:create}
\end{listing}


O código apresentado na Listagem \ref{list:bulk:create} ilustra o procedimento de criação de ofertas em lote. Inicialmente, realiza-se uma busca pelas disciplinas selecionadas, sejam elas opcionais ou obrigatórias, seguida por uma filtragem das disciplinas já cadastradas, evitando-se, assim, duplicidades. Em seguida, a operação de salvamento em lote é executada no banco de dados por meio do repositório.

\subsubsection{Avaliação de Planos de Ensino}

A página tem como objetivo apresentar os planos de ensino preenchidos pelo professor, conforme descrito na Sessão \ref{subsec:constr:professor}, para avaliação pelos Membros do Colegiado. Nessa página, é exibida uma lista de disciplinas cujos planos já foram preenchidos. Ao selecionar uma disciplina, o respectivo plano é apresentado ao usuário, juntamente com as opções de aprovar ou rejeitar \ref{fig:avaliacao:plano}. No caso de rejeição, é necessário fornecer uma justificativa para concluir a ação.

% \begin{figure}[h]
%     \centering
%     \includegraphics[scale=0.25]{imgs/avaliação.png}
%     \caption{Tela avaliação de plano}
%     \label{fig:avaliacao:plano}
% \end{figure}



\subsection{Funcionalidades do Departamento} \label{subsec:constr:departamento}

As funcionalidades do Departamento \ref{fig:departamento} assemelham-se amplamente às apresentadas para o Curso na Sessão \ref{subsec:constr:curso}. Assim como ocorre no caso do Curso, ao ser criado, o Departamento passa a ser gerido pelo Chefe do Departamento indicado no momento de sua criação, a quem são atribuídas permissões de Administrador de Departamento no sistema. As semelhanças estendem-se também às telas de "Informação" e "Professores", que operam de maneira análoga às seções Informações e Membros do Colegiado do Curso, porém exibindo dados específicos do Departamento selecionado e oferecendo a funcionalidade de adicionar professores que poderão solicitar a ministração das ofertas criadas pelo Curso. Destaca-se ainda que, de forma semelhante ao Administrador de Curso e ao Administrador de Departamento, ao vincular um professor ao Departamento, este receberá automaticamente as permissões de professor no sistema, decorrentes do vínculo estabelecido.

% \begin{figure}[h]
%     \centering
%     \includegraphics[scale=0.25]{imgs/content-departamento.png}
%     \caption{Conteúdo do Departamento}
%     \label{fig:departamento}
% \end{figure}

\subsubsection{Encargos}\label{subsec:dpto:encargos}

Esta tela é responsável pelo gerenciamento dos encargos semestrais do Departamento. Nela são listados todos os encargos relativos às disciplinas ofertadas pelos cursos, bem como suas respectivas informações. Nessa fase, espera-se que os professores se candidatem para ministrar as disciplinas; uma vez que manifestem interesse, suas solicitações serão exibidas na lista conforme demonstrado na Figura \ref{fig:encardo:dpto}. O Chefe do Departamento possui a prerrogativa de excluir solicitações, permitindo a seleção final do professor. Após a escolha, o Chefe pode bloquear qualquer edição adicional por parte dos professores, utilizando o ícone localizado acima da listagem. Por fim, a tela oferece a opção de gerar uma mensagem para oficializar a seleção dos professores ao curso, recurso semelhante ao apresentado para o curso na Sessão \ref{subsec:oferta:curso}, a Listagem \ref{list:text:curse} demonstra como o texto é gerado a partir das ofertas e professores selecionados.


% \begin{figure}[h]
%     \centering
%     \includegraphics[scale=0.25]{imgs/encargos.png}
%     \caption{Encargos do departamento}
%     \label{fig:encardo:dpto}
% \end{figure}


\begin{listing}[h]
    \begin{minted}[linenos, bgcolor=lightgray]{bash}
<div>
    {offersByCourse && selectedCourse && (
      <div>
        {`Prezado(a) Coordenador(a), ${
          offersByCourse[selectedCourse].course.admins.find(
            (e) => e.adminRole === CourseAdminRole.COORDINATOR
          ).user.name
        }`}
        {offersByCourse[selectedCourse]?.offers.map((e, idx) => (
          <>
            <br />
            <br />
            <span>
              {`O(s) professor(es) ${e.teachers
                .map((eachTeacher) => eachTeacher.name)
                .join(", ")} lecionará(ão) a disciplina ${
                e.subject.name
              }, CH ${
                e.subject.workload
              }, para o curso ${selectedCourse}. Contato ${
                e.teachers.length > 1
                  ? "dos(as) professores(as)"
                  : "do(a) professor(a)"
              }
              : ${e.teachers.map((each) => each.email).join(", ")}`}
            </span>
          </>
        ))}
        <br />
        <br />
        Atenciosamente,
        <br />
        {user.name}
</div>    
    \end{minted}
    \caption{Código geração de mensagem para Curso}
    \label{list:text:curse}
\end{listing}


\subsection{Funcionalidades do Professor} \label{subsec:constr:professor}

Ao acessar o sistema, o usuário com perfil de Professor terá acesso aos módulos de Departamento e Minhas Disciplinas. Conforme apresentado na Seção \ref{subsec:dpto:encargos}, o Professor poderá se candidatar para ministrar disciplinas, sendo-lhe apresentada uma opção de vínculo ao acessar a tela de Encargos do Departamento. A Figura \ref{fig:prof:encargo} ilustra este comportamento. Após o encerramento das edições realizado pelo Chefe de Departamento, as disciplinas atribuídas aos professores serão exibidas no módulo Minhas Disciplinas. Nessa etapa, o professor torna-se responsável pelo preenchimento dos planos de ensino, que serão posteriormente submetidos à aprovação dos membros do colegiado. Conforme ilustrado na Figura \ref{fig:prof:minhas}, ao selecionar uma disciplina, o professor poderá associar seu plano de ensino à oferta correspondente.

% \begin{figure}[h]
%     \centering
%     \includegraphics[scale=0.25]{imgs/professor-encargo.png}
%     \caption{Vinculo professor e encargo}
%     \label{fig:prof:encargo}
% \end{figure}


% \begin{figure}[h]
%     \centering
%     \includegraphics[scale=0.25]{imgs/tela-professor.png}
%     \caption{Modulo Minhas Disciplinas}
%     \label{fig:prof:minhas}
% \end{figure}


\newpage
\section{Avaliação do Sistema}\label{ref:avaliação}

Para avaliar se o trabalho alcançou os objetivos apresentados no Capitulo \ref{cap:intro}, foi proposta uma análise do sistema desenvolvido, realizada por usuários que poderiam utilizá-lo em suas atividades semestrais. Para esse propósito, foi inicialmente elaborado um breve tutorial apresentando o sistema e suas funcionalidades por meio de vídeos. A estratégia adotada consistiu na criação de filmagens segmentadas de acordo com os diferentes perfis de usuários, considerando suas respectivas funções.

Esses vídeos foram compartilhados com o secretário do curso de Ciência da Computação da Universidade Federal de São João del Rei, bem como com a secretária do Departamento do mesmo curso e instituição. Além disso, foram disponibilizados usuários fictícios e uma base de dados de teste, permitindo que os participantes experimentassem as funcionalidades do sistema e compreendessem como ele poderia ser utilizado em suas atividades.

Por fim, para fins de avaliação, foi disponibilizado um questionário com as seguintes afirmações:

\begin{itemize}
    \item Eu acho que gostaria de usar esse sistema com frequência.
    \item Eu acho o sistema desnecessariamente complexo.
    \item Eu acho que precisaria de ajuda de uma pessoa com conhecimentos técnicos para usar o sistema.
    \item Eu acho que as várias funções do sistema estão muito bem integradas.
    \item Eu acho que o sistema apresenta muita inconsistência.
    \item Eu imagino que as pessoas aprenderão como usar esse sistema rapidamente.
    \item Eu achei o sistema atrapalhado de usar.
    \item Eu precisei aprender várias coisas novas antes de conseguir usar o sistema.
    \item Você acha que o sistema fez você cumprir suas atividades mais rapidamente?
    \item Você acha que o sistema ajuda ou atrapalha a cumprir suas atividades?
\end{itemize}

Para cada afirmação, foi estabelecida uma escala de respostas de 1 a 10, em que 1 corresponde a "Discordo Totalmente" e 10 a "Concordo Totalmente". As únicas exceções foram as últimas duas perguntas, cuja resposta poderia ser escolhida entre as opções "Sim" ou "Não".

Por fim, foi incluída uma última pergunta aos usuários com o objetivo de coletar \textit{feedbacks} gerais e sugestões para possíveis implementações futuras. Nesse contexto, foi questionado: "Sentiu falta de alguma funcionalidade? Caso sim, qual?". A resposta para essa pergunta foi disponibilizada em formato de texto livre, permitindo que os usuários compartilhassem suas impressões e comentários sobre o sistema de forma espontânea.


\subsection{Resultados obtidos}\label{ref:avaliação}

Com base no questionário aplicado, foram obtidas as seguintes respostas.

\begin{table}[h]
\begin{tabular}{|l|c|}
\hline
\rowcolor[HTML]{96FFFB} 
Afirmação/Pergunta                                                                                                                       & \multicolumn{1}{l|}{\cellcolor[HTML]{96FFFB}Média das Respostas} \\ \hline
Eu acho que gostaria de usar esse sistema com frequência.                                                                                & 10                                                           \\ \hline
Eu acho o sistema desnecessariamente complexo.                                                                                           & 1                                                            \\ \hline
\begin{tabular}[c]{@{}l@{}}Eu acho que precisaria de ajuda de uma pessoa \\ com conhecimentos técnicos para usar o sistema.\end{tabular} & 1                                                            \\ \hline
\begin{tabular}[c]{@{}l@{}}Eu acho que as várias funções do sistema estão muito \\ bem integradas.\end{tabular}                          & 10                                                           \\ \hline
Eu acho que o sistema apresenta muita inconsistência.                                                                                    & 1                                                            \\ \hline
\begin{tabular}[c]{@{}l@{}}Eu imagino que as pessoas aprenderão como usar esse \\ sistema rapidamente.\end{tabular}                      & 10                                                           \\ \hline
Eu achei o sistema atrapalhado de usar.                                                                                                  & 1                                                            \\ \hline
\begin{tabular}[c]{@{}l@{}}Eu precisei aprender várias coisas novas antes de conseguir \\ usar o sistema.\end{tabular}                   & 1                                                            \\ \hline
\begin{tabular}[c]{@{}l@{}}Você acha que o sistema fez você cumprir suas atividades \\ mais rapidamente?\end{tabular}                    & 1 resposta Sim                                                          \\ \hline
\begin{tabular}[c]{@{}l@{}}Você acha que o sistema ajuda ou atrapalha a cumprir \\ suas atividades?\end{tabular}                         & 1 resposta Sim                                                          \\ \hline
\end{tabular}
\end{table}


Quanto à pergunta aberta "Sentiu falta de alguma funcionalidade? Em caso afirmativo, qual?", foi registrada a seguinte resposta:

\begin{quote}
O sistema quando testado por nós não estava com todas as funcionalidades, achei interessante e bem interativo o manuseio do mesmo. Foi feito os testes necessários e foi muito positivo. 

Em relação a essa pergunta "Você acha que o sistema ajuda ou atrapalha a cumprir suas atividades?". A resposta foi sim, mas acredito que deveria ser Ajuda ou atrapalha nas respostas. Na minha concepção o sistema agrega e ajuda aos trabalhos administrativos das coordenadorias de curso. \end{quote}

Nesse contexto, as funcionalidades que estavam ausentes foram implementadas na etapa final do trabalho, e os participantes dos testes foram convidados a avaliar a nova funcionalidade.

Assim, com base nos resultados obtidos a partir das respostas ao questionário, foi possível verificar que o software alcançou seus principais objetivos. As percepções dos usuários foram amplamente positivas, confirmando que se trata de uma solução com elevado nível de usabilidade e cujas funcionalidades efetivamente atenderam às necessidades do público-alvo.
% 
\chapter{Results}\label{cap:results}
This chapter presents the main results obtained from the development and evaluation of the project.

\section{Overview of Results}
Provide a summary of the main results achieved and their relevance to the project objectives.

\section{System Demonstration}
Describe and illustrate the main features of the system, including screenshots or diagrams if applicable.

\section{Performance Evaluation}
Present any performance tests, benchmarks, or analyses conducted to assess the system's efficiency and effectiveness.

\section{User Feedback}
Summarize feedback received from users or stakeholders, highlighting strengths and areas for improvement.

\section{Discussion}
Discuss the significance of the results, comparing them with initial expectations and related work.

\chapter{Conclusão}\label{cap:conclusao}

Durante o desenvolvimento desde trabalho, foi possível constatar a complexidade e a extensão do processo semestral realizado pelas coordenações de curso. Nesse contexto, a implementação de sistemas informatizados assume um papel fundamental na modernização e otimização das atividades gerenciais, contribuindo para a melhoria da comunicação interna e externa entre os setores, a agilidade e padronização das tarefas, além de minimizar a ocorrência de retrabalhos.

\section{Contribuições}

O trabalho desenvolvido resultou em uma ferramenta que centraliza, de forma eficiente, o processo de registro de disciplinas e a alocação de encargos didáticos em um único sistema. Com isso, oferece suporte às coordenações de curso e departamentos na organização e no planejamento dos encargos didáticos a cada período letivo. Além disso, simplifica o processo de definição das disciplinas a serem ofertadas em cada período, proporcionando aos departamentos uma visão clara e detalhada de suas responsabilidades. Por fim, o sistema busca agilizar e otimizar a indicação de docentes para os encargos atribuídos, tornando esse processo mais eficiente e assertivo.


\section{Limitações}\label{sec:limitacoes}

Embora o trabalho apresentado tenha alcançado os objetivos propostos, durante o processo de desenvolvimento foram identificadas possíveis limitações em seu funcionamento. O estudo foi conduzido com foco específico na aplicação ao curso de Ciência da Computação da Universidade Federal de São João del Rei, o que pode resultar em restrições caso seja adaptado para outros cursos ou instituições. Contudo, o sistema foi projetado de maneira a ser flexível, permitindo futuras implementações e customizações relacionadas às suas regras de negócio.


\section{Trabalhos Futuros}

Por fim, com o objetivo de fomentar a continuidade deste estudo, foram sugeridas linhas de trabalho futuro, identificadas a partir das necessidades observadas durante a aplicação do sistema em situações reais.

Conforme apresentado na Sessão \ref{sec:limitacoes}, a possibilidade de novas implementações, considerando as demandas de outros cursos e instituições, constitui uma oportunidade para ampliar a escala e a abrangência do sistema.

Adicionalmente, a inclusão de módulos destinados ao gerenciamento de atividades como a elaboração de horários, considerando a disponibilidade de professores, e o planejamento do uso de salas e laboratórios, representa outro conjunto de funcionalidades semestrais que poderiam enriquecer o sistema. Tais melhorias, especialmente com a aplicação de técnicas de Inteligência Artificial, têm o potencial de tornar o software ainda mais robusto e completo.

% ----------------------------------------------------------
% -- POST-TEXTUAL ELEMENTS
% ----------------------------------------------------------
\postextual
\bookmarksetup{startatroot} % Resets bookmark hierarchy for post-textual elements

% --- Bibliography ---
% The 'abnt-num' style creates numeric citations. For author-date, use 'abnt-alf'.
\bibliographystyle{abnt-num}
% The .bib file containing your references
% [TODO: Add your references to the 'references.bib' file]
\bibliography{references}

% --- Appendices (Optional) ---
% [TODO: Uncomment if you have appendices]

\chapter*{Appendices}
\hypertarget{appendices}{}  % Create the target for the hyperlink
\addunnumberedchapter[appendices]{Appendices}

% Change section numbering to use letters
\renewcommand{\thesection}{\Alph{section}}


Content here...

\section{First Appendix}
\section{Second Appendix}
\section{Third Appendix}

% --- Attachments (Optional) ---
% [TODO: Uncomment if you have attachments]
\chapter*{Anexos}
\hypertarget{anexos}{}  % Create the target for the hyperlink
\addunnumberedchapter[anexos]{Anexos}

% Change section numbering to use letters and restart from A
\setcounter{section}{0}
\renewcommand{\thesection}{\Alph{section}}

Content here...

\section{Primeiro Anexo}
\section{Segundo Anexo}
\section{Terceiro Anexo}

% --- Index (Optional) ---
% [TODO: Uncomment to print the remissive index]
% \printindex
% --- ADD THE CLOSING PAGE COMMAND HERE ---
\imprimirclosingpage
\end{document}