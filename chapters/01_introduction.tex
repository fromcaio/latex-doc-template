\chapter{Introdução}\label{cap:intro}

\section{Contextualização}
A disseminação de conhecimento gratuito tem se mostrado uma prática cada vez mais relevante no contexto educacional, especialmente com o crescimento de plataformas online e de iniciativas de educação aberta. Professores, pesquisadores e outros produtores de conteúdo — especialmente nas ciências exatas, como ciência da computação, matemática e física — enfrentam, no entanto, diversas dificuldades ao tentar compartilhar materiais ricos, interativos e bem formatados de forma acessível e sem barreiras tecnológicas significativas. A produção e publicação de conteúdos como cursos e artigos frequentemente exige o domínio de múltiplas ferramentas, além de conhecimento técnico para lidar com formatações, códigos, fórmulas matemáticas e outros elementos fundamentais ao ensino moderno.

% \begin{listing}[h]
%     \lstinputlisting[language=c, firstline=10, lastline=20]{../../src/shell.c}
%     \caption{Trecho de código em c extraído do arquivo \texttt{/src/shell.c.}}
%     \label{list:label}
% \end{listing}


Apesar do avanço de editores e plataformas voltadas para a escrita digital, muitos desses ambientes não foram concebidos para atender às necessidades específicas de educadores que desejam compartilhar materiais com qualidade técnica e visual, mas sem abrir mão da simplicidade de uso. A ausência de funcionalidades integradas — como realce automático de código (code highlighting), suporte facilitado a fórmulas matemáticas escritas em LaTeX, criação de tabelas acessíveis e estruturação semântica do conteúdo — torna o processo de produção mais demorado e propenso a erros.

Nesse cenário, foi desenvolvido o FromCaio, um sistema web interativo voltado à criação e publicação de cursos e artigos educacionais. A proposta central é oferecer uma plataforma intuitiva que reduza as barreiras técnicas enfrentadas por educadores, promovendo a criação de conteúdo gratuito, acessível e com recursos avançados de formatação. O sistema permite que os próprios autores construam e publiquem seus materiais diretamente na plataforma, com o suporte de funcionalidades modernas que otimizam a experiência de escrita e de leitura.

Nesse cenário, foi desenvolvido o FromCaio, um sistema web interativo voltado à criação e publicação de cursos e artigos educacionais. A proposta central é oferecer uma plataforma intuitiva que reduza as barreiras técnicas enfrentadas por educadores, promovendo a criação de conteúdo gratuito, acessível e com recursos avançados de formatação. O sistema permite que os próprios autores construam e publiquem seus materiais diretamente na plataforma, contando com funcionalidades modernas que otimizam tanto a experiência de escrita quanto a de leitura.


% citações citacao1\cite{pressman2021} citacao2 \cite{KeRainerJr2019}.

\section{Motivação}
A educação tem o poder de transformar vidas. Essa afirmação, embora frequentemente repetida, se comprova de forma concreta na trajetória de inúmeros indivíduos — inclusive na do próprio autor deste trabalho. O ensino revelou-se um divisor de águas, não apenas por despertar o desejo de aprender e compartilhar conhecimento, mas também por dar sentido a aspectos da realidade que antes passavam despercebidos. Através da educação, temas antes invisíveis tornaram-se compreensíveis, e eventualmente até mesmo belos.

Aprender, apesar de desafiador, é uma das experiências mais enriquecedoras que alguém pode vivenciar. Seu impacto se manifesta de forma ampla: no plano intelectual, amplia a visão de mundo e desenvolve o pensamento crítico; no profissional, abre oportunidades e fortalece habilidades; e no pessoal, alimenta a curiosidade, o senso de propósito e a realização de contribuir com os outros.

Foi a partir dessa vivência que surgiu o desejo de colaborar ativamente para a ampliação do acesso ao conhecimento, valorizando especialmente os profissionais que se dedicam à produção científica e educacional. A motivação principal deste trabalho é criar um ambiente que respeite e reconheça esse esforço, oferecendo uma plataforma aberta, intuitiva e acessível tanto para quem ensina quanto para quem aprende.

A proposta do FromCaio nasce, portanto, da convicção de que democratizar o conhecimento vai além de simplesmente distribuí-lo. É preciso garantir sua qualidade, relevância e potencial transformador. O objetivo é proporcionar um espaço onde educadores possam produzir e compartilhar conteúdos que gerem impacto verdadeiro — tanto na vida de indivíduos quanto na atuação de empresas e instituições. Este projeto busca ser um passo concreto nessa direção.

\section{Objetivos}

\subsection{Objetivo Geral}

Desenvolver uma plataforma independente de publicação de materiais didáticos e científicos, com foco na autonomia dos autores, na permanência e portabilidade do conteúdo e no impacto profissional e social da produção intelectual.

\subsection{Objetivos Específicos}

\begin{itemize}
    \item Oferecer suporte a elementos essenciais para a produção técnico-científica, como trechos de código, expressões matemáticas e organização semântica do conteúdo;
    \item Viabilizar a publicação e a colaboração em projetos educacionais mesmo fora de contextos institucionais, eliminando barreiras técnicas e operacionais comuns em plataformas tradicionais;
    \item Adotar o formato \textit{markdown}, estendido com funcionalidades específicas da plataforma, a fim de garantir a longevidade, a portabilidade e o reaproveitamento dos materiais produzidos;
    \item Implementar recursos avançados como realce de sintaxe (\textit{code highlighting}) para múltiplas linguagens de programação;
    \item Desenvolver, em versões futuras, funcionalidades como a geração automática de gráficos interativos, ampliando as possibilidades pedagógicas e científicas;
    \item Estabelecer uma política de publicação responsável, com curadoria leve, baseada na expertise dos autores, a fim de assegurar a qualidade e a credibilidade dos conteúdos publicados.
\end{itemize}

\section{Justificativa}

A proposta deste trabalho se justifica por sua relevância em múltiplas esferas: acadêmica, educacional, social e tecnológica. Em um cenário em que o acesso ao conhecimento cresce continuamente, aumenta também a necessidade por ferramentas que não apenas permitam a publicação de conteúdo, mas que assegurem sua qualidade, acessibilidade e independência.

Apesar do avanço das plataformas digitais, ainda existem lacunas significativas — especialmente para educadores, pesquisadores e estudantes que desejam publicar ou colaborar na produção de conteúdo. Isso é ainda mais evidente entre aqueles que buscam alternativas gratuitas de publicação, sem depender de sistemas institucionais mais rígidos, os quais frequentemente impõem barreiras técnicas e operacionais. Essas limitações acabam desestimulando a criação de conteúdo educativo de qualidade — um processo que, por si só, já é desafiador — e que poderia ser significativamente fortalecido com a redução das barreiras técnicas envolvidas.


Ao contrário de ambientes virtuais de aprendizagem amplamente utilizados, como o Moodle — cuja proposta central é gerenciar cursos e atividades acadêmicas institucionais —, o \textit{FromCaio} foi concebido especificamente como uma plataforma de publicação autônoma de conteúdos didáticos e científicos. Embora o Moodle cumpra eficientemente seu papel em contextos formais de ensino, adaptá-lo por meio de extensões para atender a fluxos editoriais mais flexíveis exigiria modificações complexas e uma adesão a requisitos técnicos institucionais. O \textit{FromCaio}, por sua vez, busca justamente superar essas limitações, permite que professores e estudantes colaborem na criação de projetos, orientados institucionalmente ou não, sem que precisem se adequar a modelos rígidos de publicação, favorecendo a clareza, a didática e o controle autoral sobre o material produzido.

Essa independência é sustentada por decisões técnicas que favorecem tanto a longevidade quanto a portabilidade dos materiais produzidos. A adoção do formato \textit{markdown} — ainda que estendido com funcionalidades específicas da plataforma — permite a migração e o reaproveitamento de conteúdos de forma ágil e eficaz, inclusive a partir de outros formatos amplamente utilizados no meio acadêmico, como slides e roteiros de aula. Essa flexibilidade é ainda fortalecida pelo uso de ferramentas externas baseadas em inteligência artificial generativa, que auxiliam na conversão entre formatos distintos, preservando a coerência, a estrutura e a utilidade pedagógica do conteúdo ao longo de diferentes contextos de publicação.

Complementando esse conjunto de soluções, a plataforma oferece recursos que ampliam o potencial técnico e didático dos materiais. Um dos destaques é o suporte ao realce de sintaxe (\textit{code highlighting}) para mais de 40 linguagens de programação — incluindo C, Python, Assembly e Arduino —, número que continuará a crescer conforme novas necessidades surgirem e conteúdos forem adicionados. Versões futuras também preveem funcionalidades como a geração automática de gráficos interativos, abrindo novas possibilidades de aplicação pedagógica e científica.

Por fim, embora valorize a liberdade editorial, a proposta adota uma política de publicação responsável, pautada na credibilidade e na qualidade. A autoria é reservada a usuários com expertise comprovada — como professores universitários, profissionais especializados ou estudantes sob orientação adequada. Essa curadoria leve tem como objetivo assegurar que os conteúdos publicados estejam alinhados ao propósito da plataforma: promover impacto real e positivo na formação intelectual, profissional e pessoal dos leitores.

\section{Organização da Monografia}

Além deste capítulo, os demais estão organizados da seguinte forma:

\begin{itemize}
    \item Capítulo \ref{cap:ref_teorico}: Contém o referencial teórico, onde foram fundamentados os conceitos, técnicas e tecnologias utilizadas no decorrer do estudo. Além disso, também são apresentados os trabalhos relacionados;
    \item Capítulo \ref{cap:metodologia}: Neste capitulo é apresentado a metodologia, contendo as etapas realizadas no desenvolvimento do sistema;
    \item Capítulo \ref{cap:dev_sist}: São relatados as etapas da construção dos sistema, assim como os resultados alcançados no desenvolvimento e testes;
    \item Capítulo \ref{cap:conclusao}: Neste capitulo são apresentadas as conclusões dos autores junto as considerações finais e proposta de trabalhos futuros;
\end{itemize}